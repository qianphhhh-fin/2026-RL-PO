%%%%%%%% ICML 2025 EXAMPLE LATEX SUBMISSION FILE %%%%%%%%%%%%%%%%%

\documentclass{article}

% Recommended, but optional, packages for figures and better typesetting:
\usepackage{microtype}
\usepackage{graphicx}
\usepackage{subfigure}
\usepackage{booktabs} % for professional tables

% hyperref makes hyperlinks in the resulting PDF.
% If your build breaks (sometimes temporarily if a hyperlink spans a page)
% please comment out the following usepackage line and replace
% \usepackage{icml2025} with \usepackage[nohyperref]{icml2025} above.
\usepackage{hyperref}


% Attempt to make hyperref and algorithmic work together better:
\newcommand{\theHalgorithm}{\arabic{algorithm}}

% Use the following line for the initial blind version submitted for review:
% \usepackage{icml2025}

% If accepted, instead use the following line for the camera-ready submission:
\usepackage[accepted]{icml2025}

% For theorems and such
\usepackage{amsmath}
\usepackage{amssymb}
\usepackage{mathtools}
\usepackage{amsthm}

% if you use cleveref..
\usepackage[capitalize,noabbrev]{cleveref}

%%%%%%%%%%%%%%%%%%%%%%%%%%%%%%%%
% THEOREMS
%%%%%%%%%%%%%%%%%%%%%%%%%%%%%%%%
\theoremstyle{plain}
\newtheorem{theorem}{Theorem}[section]
\newtheorem{proposition}[theorem]{Proposition}
\newtheorem{lemma}[theorem]{Lemma}
\newtheorem{corollary}[theorem]{Corollary}
\theoremstyle{definition}
\newtheorem{definition}[theorem]{Definition}
\newtheorem{assumption}[theorem]{Assumption}
\theoremstyle{remark}
\newtheorem{remark}[theorem]{Remark}

% Todonotes is useful during development; simply uncomment the next line
%    and comment out the line below the next line to turn off comments
%\usepackage[disable,textsize=tiny]{todonotes}
\usepackage[textsize=tiny]{todonotes}


% The \icmltitle you define below is probably too long as a header.
% Therefore, a short form for the running title is supplied here:
\icmltitlerunning{Submission and Formatting Instructions for ICML 2025}

\begin{document}

\twocolumn[
\icmltitle{Differentiable Multi-Period Portfolio Optimization (DMPO)}

% It is OKAY to include author information, even for blind
% submissions: the style file will automatically remove it for you
% unless you've provided the [accepted] option to the icml2025
% package.

% List of affiliations: The first argument should be a (short)
% identifier you will use later to specify author affiliations
% Academic affiliations should list Department, University, City, Region, Country
% Industry affiliations should list Company, City, Region, Country

% You can specify symbols, otherwise they are numbered in order.
% Ideally, you should not use this facility. Affiliations will be numbered
% in order of appearance and this is the preferred way.
\icmlsetsymbol{equal}{*}

\begin{icmlauthorlist}
\icmlauthor{Firstname1 Lastname1}{equal,yyy}
\icmlauthor{Firstname2 Lastname2}{equal,yyy,comp}
\icmlauthor{Firstname3 Lastname3}{comp}
\icmlauthor{Firstname4 Lastname4}{sch}
\icmlauthor{Firstname5 Lastname5}{yyy}
\icmlauthor{Firstname6 Lastname6}{sch,yyy,comp}
\icmlauthor{Firstname7 Lastname7}{comp}
%\icmlauthor{}{sch}
\icmlauthor{Firstname8 Lastname8}{sch}
\icmlauthor{Firstname8 Lastname8}{yyy,comp}
%\icmlauthor{}{sch}
%\icmlauthor{}{sch}
\end{icmlauthorlist}

\icmlaffiliation{yyy}{Department of XXX, University of YYY, Location, Country}
\icmlaffiliation{comp}{Company Name, Location, Country}
\icmlaffiliation{sch}{School of ZZZ, Institute of WWW, Location, Country}

\icmlcorrespondingauthor{Firstname1 Lastname1}{first1.last1@xxx.edu}
\icmlcorrespondingauthor{Firstname2 Lastname2}{first2.last2@www.uk}

% You may provide any keywords that you
% find helpful for describing your paper; these are used to populate
% the "keywords" metadata in the PDF but will not be shown in the document
\icmlkeywords{Machine Learning, ICML}

\vskip 0.3in
]

% this must go after the closing bracket ] following \twocolumn[ ...

% This command actually creates the footnote in the first column
% listing the affiliations and the copyright notice.
% The command takes one argument, which is text to display at the start of the footnote.
% The \icmlEqualContribution command is standard text for equal contribution.
% Remove it (just {}) if you do not need this facility.

%\printAffiliationsAndNotice{}  % leave blank if no need to mention equal contribution
\printAffiliationsAndNotice{\icmlEqualContribution} % otherwise use the standard text.

\begin{abstract}
This document provides a basic paper template and submission guidelines.
Abstracts must be a single paragraph, ideally between 4--6 sentences long.
Gross violations will trigger corrections at the camera-ready phase.
\end{abstract}
%%%%%%%%%%%%%%%%%%%%%%%%%%%%%%%%%%%%%%%%%%%%%%%%%%%%%%%%%%%%%%%%%%%%%%%%%%%%%%%
% SECTION: INTRODUCTION
%%%%%%%%%%%%%%%%%%%%%%%%%%%%%%%%%%%%%%%%%%%%%%%%%%%%%%%%%%%%%%%%%%%%%%%%%%%%%%%

\section{Introduction}
\label{sec:intro}

Dynamic portfolio optimization presents a fundamental challenge in quantitative finance, requiring agents to balance expected returns against risks and transaction costs in a stochastic environment. While classical Mean-Variance frameworks \cite{markowitz1952} provide theoretical foundations, they often lack the flexibility to adapt to changing market regimes. Conversely, deep Reinforcement Learning (RL) has emerged as a powerful tool for learning complex, path-dependent strategies directly from data \cite{jiang2017deep}. However, a critical limitation of standard model-free RL algorithms (e.g., PPO, DDPG) is their inability to strictly enforce hard constraints—such as turnover limits or cardinality constraints—during both exploration and deployment. Conventional approaches typically rely on reward shaping (soft penalties), which offers no guarantees of feasibility and often leads to suboptimal convergence due to conflicting objectives.

To bridge this gap, we introduce \textbf{Differentiable Multi-Period Portfolio Optimization (DMPO)}, a hybrid framework that integrates deep sequence modeling with differentiable convex optimization. Unlike standard RL agents that directly output portfolio weights, DMPO learns to generate a \textit{market view} vector, which parameterizes a quadratic programming (QP) layer embedded within the policy network. By leveraging implicit differentiation through the Karush-Kuhn-Tucker (KKT) conditions, our method permits end-to-end training via gradient ascent while guaranteeing rigorous satisfaction of turnover constraints at every time step. Our extensive experiments demonstrate that DMPO achieves zero constraint violations and superior risk-adjusted returns compared to traditional benchmarks.

%%%%%%%%%%%%%%%%%%%%%%%%%%%%%%%%%%%%%%%%%%%%%%%%%%%%%%%%%%%%%%%%%%%%%%%%%%%%%%%
% SECTION: METHODOLOGY
%%%%%%%%%%%%%%%%%%%%%%%%%%%%%%%%%%%%%%%%%%%%%%%%%%%%%%%%%%%%%%%%%%%%%%%%%%%%%%%

\section{Methodology}
\label{sec:method}

We formulate the portfolio management problem as a constrained Markov Decision Process (CMDP) and propose a neural architecture that explicitly embeds a convex optimization solver as the final layer of the policy network.

\subsection{Problem Formulation}

Consider a financial market with $N$ assets over discrete time steps $t = 0, \dots, T$. Let $p_t \in \mathbb{R}^N_+$ denote the asset prices and $r_t \in \mathbb{R}^N$ denote the vector of asset returns at time $t$. The portfolio state is represented by the allocation vector $w_t \in \Delta^{N}$, where $\Delta^{N} = \{w \in \mathbb{R}^N \mid \mathbf{1}^\top w = 1, w \succeq 0\}$ is the standard simplex.

We define the state space $\mathcal{S}$ as a tuple $s_t = (X_t, w_{t-1})$, where $X_t \in \mathbb{R}^{k \times N}$ represents the trailing window of historical market features (e.g., returns) of length $k$, and $w_{t-1}$ is the portfolio weight vector from the previous step. The action $a_t$ corresponds to the new portfolio weights $w_t$.

The agent aims to maximize the cumulative risk-adjusted return subject to a hard turnover constraint. The turnover $\mathcal{T}_t$ is defined as the $L_1$-norm of the rebalancing vector:
\begin{equation}
    \mathcal{T}_t = \| w_t - w_{t-1} \|_1.
\end{equation}
Crucially, we impose a strict limit $\delta$ on turnover (e.g., $\delta=0.1$) to control transaction costs and market impact.

\subsection{The DMPO Architecture}

The core innovation of DMPO is the decoupling of \textit{market view generation} (learned via Neural Networks) and \textit{constraint enforcement} (solved via Convex Optimization). The policy $\pi_\theta$ consists of two distinct modules:

\subsubsection{1. View Generation Network}
We employ a Long Short-Term Memory (LSTM) network to extract temporal dependencies from the market history $X_t$. The network outputs a latent representation, which is then projected to a "view" vector $\hat{\mu}_t \in \mathbb{R}^N$:
\begin{align}
    h_t &= \text{LSTM}(X_t; \theta_{\text{enc}}), \\
    \hat{\mu}_t &= \text{Linear}(h_t \oplus \phi(w_{t-1}); \theta_{\text{head}}),
\end{align}
where $\oplus$ denotes concatenation and $\phi(\cdot)$ is a feature embedding of the current position. Unlike traditional expected return vectors, $\hat{\mu}_t$ is an unconstrained latent parameter learned end-to-end to maximize the RL objective.

\subsubsection{2. Differentiable QP Layer}
Instead of treating $\hat{\mu}_t$ as the action, we feed it into a differentiable optimization layer. The final action $w_t$ is obtained by solving the following Quadratic Program (QP):
\begin{subequations}
\label{eq:qp_layer}
\begin{align}
    w_t^* = \mathop{\arg\min}_{w \in \mathbb{R}^N} \quad & \frac{1}{2} \|w\|_2^2 - \hat{\mu}_t^\top w \label{eq:obj} \\
    \text{s.t.} \quad & \mathbf{1}^\top w = 1, \\
    & w \succeq 0, \\
    & \| w - w_{t-1} \|_1 \leq \delta. \label{eq:turnover_const}
\end{align}
\end{subequations}
Here, the objective function \eqref{eq:obj} encourages the portfolio to align with the learned signal $\hat{\mu}_t$ while the regularization term $\frac{1}{2}\|w\|_2^2$ prevents extreme concentration and ensures the problem is strictly convex. The constraint \eqref{eq:turnover_const} strictly enforces the turnover limit.



\subsection{End-to-End Learning via Implicit Differentiation}

The optimization problem \eqref{eq:qp_layer} defines an implicit mapping $w_t^* = \mathcal{P}(\hat{\mu}_t, w_{t-1})$. To train the upstream parameters $\theta$ using standard policy gradient methods (e.g., PPO), we require the Jacobian $\partial w_t^* / \partial \hat{\mu}_t$.

Since $w_t^*$ is the solution to a convex QP, it satisfies the KKT optimality conditions $F(z^*, \psi) = 0$, where $z^*$ contains the primal and dual optimal variables, and $\psi = \{\hat{\mu}_t, w_{t-1}\}$ are the layer parameters. By the Implicit Function Theorem, we compute the gradients by solving a linear system involving the KKT matrix:
\begin{equation}
    \frac{\partial z^*}{\partial \psi} = - \left( \frac{\partial F}{\partial z^*} \right)^{-1} \frac{\partial F}{\partial \psi}.
\end{equation}
In practice, we utilize efficient batched solvers (e.g., OSQP) combined with the \texttt{cvxpylayers} framework to perform this backward pass efficiently on GPUs. This ensures that the RL agent receives meaningful gradients indicating how to adjust the view $\hat{\mu}_t$ to achieve better long-term rewards while respecting physical constraints.

During training, we employ Proximal Policy Optimization (PPO). To ensure exploration within the feasible region, we apply parameter-space noise to $\hat{\mu}_t$ prior to the QP layer, rather than adding noise to the output weights $w_t$. This guarantees that every exploratory action taken by the agent remains strictly feasible.


% In the unusual situation where you want a paper to appear in the
% references without citing it in the main text, use \nocite
% \nocite{langley00}

\bibliography{dmpo}
\bibliographystyle{icml2025}


%%%%%%%%%%%%%%%%%%%%%%%%%%%%%%%%%%%%%%%%%%%%%%%%%%%%%%%%%%%%%%%%%%%%%%%%%%%%%%%
%%%%%%%%%%%%%%%%%%%%%%%%%%%%%%%%%%%%%%%%%%%%%%%%%%%%%%%%%%%%%%%%%%%%%%%%%%%%%%%
% APPENDIX
%%%%%%%%%%%%%%%%%%%%%%%%%%%%%%%%%%%%%%%%%%%%%%%%%%%%%%%%%%%%%%%%%%%%%%%%%%%%%%%
%%%%%%%%%%%%%%%%%%%%%%%%%%%%%%%%%%%%%%%%%%%%%%%%%%%%%%%%%%%%%%%%%%%%%%%%%%%%%%%
\newpage
\appendix
\onecolumn
\section{You \emph{can} have an appendix here.}

You can have as much text here as you want. The main body must be at most $8$ pages long.
For the final version, one more page can be added.
If you want, you can use an appendix like this one.  

The $\mathtt{\backslash onecolumn}$ command above can be kept in place if you prefer a one-column appendix, or can be removed if you prefer a two-column appendix.  Apart from this possible change, the style (font size, spacing, margins, page numbering, etc.) should be kept the same as the main body.
%%%%%%%%%%%%%%%%%%%%%%%%%%%%%%%%%%%%%%%%%%%%%%%%%%%%%%%%%%%%%%%%%%%%%%%%%%%%%%%
%%%%%%%%%%%%%%%%%%%%%%%%%%%%%%%%%%%%%%%%%%%%%%%%%%%%%%%%%%%%%%%%%%%%%%%%%%%%%%%


\end{document}


% This document was modified from the file originally made available by
% Pat Langley and Andrea Danyluk for ICML-2K. This version was created
% by Iain Murray in 2018, and modified by Alexandre Bouchard in
% 2019 and 2021 and by Csaba Szepesvari, Gang Niu and Sivan Sabato in 2022.
% Modified again in 2023 and 2024 by Sivan Sabato and Jonathan Scarlett.
% Previous contributors include Dan Roy, Lise Getoor and Tobias
% Scheffer, which was slightly modified from the 2010 version by
% Thorsten Joachims & Johannes Fuernkranz, slightly modified from the
% 2009 version by Kiri Wagstaff and Sam Roweis's 2008 version, which is
% slightly modified from Prasad Tadepalli's 2007 version which is a
% lightly changed version of the previous year's version by Andrew
% Moore, which was in turn edited from those of Kristian Kersting and
% Codrina Lauth. Alex Smola contributed to the algorithmic style files.
